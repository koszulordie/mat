\documentclass[]{book}

%These tell TeX which packages to use.
\usepackage{array,epsfig}
\usepackage{amsmath}
\usepackage{amsfonts}
\usepackage{amssymb}
\usepackage{amsxtra}
\usepackage{amsthm}
\usepackage{mathrsfs}
\usepackage{color}


%Here I define some theorem styles and shortcut commands for symbols I use often
\theoremstyle{definition}
\newtheorem{defn}{Definition}
\newtheorem{thm}{Theorem}
\newtheorem{cor}{Corollary}
\newtheorem*{rmk}{Remark}
\newtheorem{lem}{Lemma}
\newtheorem*{joke}{Joke}
\newtheorem{ex}{Example}
\newtheorem*{soln}{Solution}
\newtheorem{prop}{Proposition}

\newcommand{\lra}{\longrightarrow}
\newcommand{\ra}{\rightarrow}
\newcommand{\surj}{\twoheadrightarrow}
\newcommand{\graph}{\mathrm{graph}}
\newcommand{\rank}{\textrm{rank}}
\newcommand{\length}{\textrm{length}}
\newcommand{\bb}[1]{\mathbb{#1}}
\newcommand{\Z}{\bb{Z}}
\newcommand{\Q}{\bb{Q}}
\newcommand{\R}{\bb{R}}
\newcommand{\C}{\bb{C}}
\newcommand{\N}{\bb{N}}
\newcommand{\M}{\mathbf{M}}
\newcommand{\m}{\mathbf{m}}
\newcommand{\MM}{\mathscr{M}}
\newcommand{\HH}{\mathscr{H}}
\newcommand{\Om}{\Omega}
\newcommand{\Ho}{\in\HH(\Om)}
\newcommand{\bd}{\partial}
\newcommand{\del}{\partial}
\newcommand{\bardel}{\overline\partial}
\newcommand{\textdf}[1]{\textbf{\textsf{#1}}\index{#1}}
\newcommand{\img}{\mathrm{img}}
\newcommand{\ip}[2]{\left\langle{#1},{#2}\right\rangle}
\newcommand{\inter}[1]{\mathrm{int}{#1}}
\newcommand{\exter}[1]{\mathrm{ext}{#1}}
\newcommand{\cl}[1]{\mathrm{cl}{#1}}
\newcommand{\ds}{\displaystyle}
\newcommand{\vol}{\mathrm{vol}}
\newcommand{\cnt}{\mathrm{ct}}
\newcommand{\osc}{\mathrm{osc}}
\newcommand{\LL}{\mathbf{L}}
\newcommand{\UU}{\mathbf{U}}
\newcommand{\support}{\mathrm{support}}
\newcommand{\AND}{\;\wedge\;}
\newcommand{\OR}{\;\vee\;}
\newcommand{\Oset}{\varnothing}
\newcommand{\st}{\ni}
\newcommand{\wh}{\widehat}
\newcommand\yellow[1]{\colorbox{yellow}{#1}}
\newcommand\ans{\underline{Answer}: }

%Pagination stuff.
\setlength{\topmargin}{-.3 in}
\setlength{\oddsidemargin}{0in}
\setlength{\evensidemargin}{0in}
\setlength{\textheight}{9.in}
\setlength{\textwidth}{6.5in}
\pagestyle{empty}


\begin{document}


% ----1----


\begin{center}
\textbf{Final Test (Wednesday 15th Dec 2021)}\\
Elements of Mathematics -- Bioinformatics for Health Sciences \\
\end{center}

\vspace{0.2 cm}

\begin{enumerate}


% ----2----

\item {\bf (1 point)} Find a basis of the column space $C(A)$ of the following matrix $A$: 
\[
A = \begin{bmatrix}
1 & 2  & 1 \\
1 & 1 & -1 \\
-2 & -2 & 2 \\
\end{bmatrix}
\]

\ans

Via Gauss-Jordan elimination we can verify that $\rank (A) = 2$. In particular, if $u$, $v$ and $w$ are the columns of $A$, it turns out that $w=2v - 3u$. In this case, any two columns of $A$ constitutes a basis of $C(A)$.


\item {\bf (1 point)} Find an example of a $3\times 3$ matrix $A$ such that $\dim C(A) = 2$ and $\dim N(A^2) = 2$.

\ans

Using the fundamental theorem of linear algebra, we know that $\dim C(A) + \dim N(A) = 3$, so the first assumption implies $\dim N(A) = 1$. A typical way to think about this kind of examples is to consider the image of each vector of the canonical basis $\{e_1,e_2,e_3\}$ of 
$\mathbb{R}^3$ by the linear transformation encoded by $A$. Note that if the said linear transformation $f$ does $f(e_1) = 0$, $f(e_2) = e_1$ and $f(e_3) = e_3$, when we do the composition $f^2 = f \circ f$ of $f$ with itself, we get $f^2(e_1) = f(f(e_1)) = f(0) = 0$, $f^2(e_2) = f(f(e_2)) = f(e_1) = 0$ and $f^2(e_3) = f(f(e_3)) = f(e_3) = e_3$. The matrix $A$ that realizes the linear transformation $f$ is

\[
A = \begin{bmatrix}
0 & 1 & 0 \\
0 & 0 & 0 \\
0 & 0 & 1
\end{bmatrix}
\]

In other words, $N(A) = \textrm{span}\{e_1\}$ and $N(A^2) = \textrm{span}\{e_1, e_2\}$. 


% ----3----

\item {\bf (2 points)} Consider the following matrix:

\[
A = \begin{bmatrix}
1 & 1 & 1 \\
-1 & 1 & 2 \\
1 & 1 & 1
\end{bmatrix}
\]
\begin{itemize}
\item[a)] {\bf (1 point)} Find a basis of $N(A)$ the null space of $A$.
\item[b)] {\bf (1 point)} Find an orthonormal basis of $N(A)$.
\end{itemize}

\ans

\begin{itemize}
\item[a)] Let's apply Gauss-Jordan elimination

\[
{\begin{bmatrix}
  1 & 1  & 1 \\
  -1 & 1  & 2 \\ 
  1 & 1  & 1 \\ 
  1 & 0 & 0 \\
  0 & 1 & 0 \\
  0 & 0 & 1
  \end{bmatrix}
}
\stackrel{i}{\sim}
{\begin{bmatrix}
  1 & 0  & 0 \\
  -1 & 2 & 3 \\ 
  1 & 0 & 0 \\ 
  1 & -1 & -1 \\
  0 & 1 & 0 \\
  0 & 0 & 1
  \end{bmatrix}
}
\stackrel{ii}{\sim}
{\begin{bmatrix}
  1 & 0  & 0 \\
  -1 & 2 & 0 \\ 
  1 & 0 & 0 \\ 
  1 & -1 & 1/2 \\
  0 & 1 & -3/2 \\
  0 & 0 & 1
  \end{bmatrix}
},
\]

From this calculation we see that $N(A) = \textrm{span}\{(1,-3, 2)\}$.

\item[b)] Note we just need to find a unit length representative of $N(A)$. If $v=(1,-3,2)$ we can pick $u = v/\length(v) = \frac{1}{\sqrt{15}} \, (1, -3, 2)$. Then $\{u\}$ is an orthonormal basis of $N(A)$.

\end{itemize}


% ----4----

\item {\bf (1 point)} Given the $2\times 2$ matrices:
\[
A = \begin{bmatrix}
1 & 0 \\
1 & 1
\end{bmatrix}\;\;\;
B = \begin{bmatrix}
1 & 2 \\
0 & 1
\end{bmatrix}
\]
Are the eigenvalues of $AB$ equal to the eigenvalues of $BA$?

\ans

Let's compute $AB$ and $BA$.

\[
AB = {\begin{bmatrix}
  1 & 2 \\
  1 & 3 
  \end{bmatrix}
}
\]

\[
BA = {\begin{bmatrix}
  3 & 2 \\
  1 & 1 
  \end{bmatrix}
}
\]

Note that $\det(AB) = \det(BA) = 1$ and $\textrm{trace}(AB) = \textrm{trace}(BA) = 4$. But the trace and determinant define the eigenvalues, as the eigenvalues of a given $2\times 2$ matrix $\Omega$ must fullfill $\lambda_1 + \lambda_2 = \textrm{trace}(\Omega)$ and $\lambda_1 \lambda_2 = \det(\Omega)$. Therefore the eigenvalues in both cases must coincide.

% ---5----

\item {\bf (1 point)} Provide the Taylor approximation of order 2 at $a=0$ of the function $f(x) = 1 / (1 + e^{-x})$.

\ans

Let's compute the first and second-order derivatives of $f(x)$:

\[
f'(x) = \frac{-1}{(1 + e^{-x})^2} \cdot (-e^{-x}) = \frac{e^{-x}}{(1+e^{-x})^2}
\]

Applying directly the product rule,

\[
f''(x) = -e^{-x} \frac{1}{(1+e^{-x})^2} + e^{-x} \frac{-2}{(1+e^{-x})^3} \cdot (-e^{-x}) = 
\frac{2e^{-2x}}{(1 + e^{-x})^3} - \frac{e^{-x}}{(1+e^{-x})^2} = 
\]
\[
= \frac{2e^{-2x}}{(1 + e^{-x})^3} - \frac{e^{-x} (1+e^{-x})}{(1+e^{-x})^3} = \frac{e^{-2x} - e^{-x}}{(1+e^{-x})^3}
\]

Alternatively, we could also observe that $f'(x) = f(x)(1 - f(x))$, then $f''(x) = f'(x)(1-f(x)) - f(x)f'(x) = f'(x) (1 - 2f(x))$. Note that both expressions coincide, since

\[
f'(x) (1 - 2f(x)) = \frac{e^{-x}}{(1 + e^{-x})^2} \frac{e^{-x} - 1}{(1+e^{-x})} = \frac{e^{-2x} - e^{-x}}{(1+e^{-x})^3}
\]

Evaluating $f$ and its first and second-order derivatives at $a=0$ gives

\[
f(0) = 1/2,\;\; f'(0) = 1/4, \;\; f''(0) = 0
\]

The Taylor approximation up to order 2 is given by a linear polynomial $T(x) = \frac{1}{2} + \frac{1}{4}x$, in other words, $f(x) = \frac{1}{2} + \frac{1}{4}x + o(x^2)$ about the point $a=0$.


% ----6----

\item {\bf (2 points)} Consider the following function:

\[
f(x,y) = e^{-(ax^2+by^2)}
\]

where $a,b\in\mathbb{R}$ are parameters of the function. Compute $\nabla f (1,1)$, i.e., the gradient vector of $f$ at the point (1,1).

\ans

Let's compute partial derivatives:

\[
\frac{\partial f}{\partial x} =  e^{-(ax^2+by^2)} (-2ax) = -2axe^{-(ax^2+by^2)}
\]
\[
\frac{\partial f}{\partial y} =  e^{-(ax^2+by^2)} (-2by) = -2bye^{-(ax^2+by^2)}
\]

Whence  $\nabla f (1,1) = (-2ae^{-(a+b)}, -2be^{-(a+b)})$

% ----7----

\item {\bf (2 points)} Determine the nature of all the critical points of the function
\[
f(x, y) = x^3 - x + y^3 - y.
\]

\ans

Let's compute partial derivatives:

\[
\frac{\partial f}{\partial x} =  3x^2 - 1, \;\;\; \frac{\partial f}{\partial y} =  3y^2 - 1
\]
\[
\frac{\partial^2 f}{\partial x^2} =  6x, \;\;\; \frac{\partial^2 f}{\partial y^2} =  6y, \;\;\; \frac{\partial^2 f}{\partial x \partial y} = 0
\]

The function has four critical points, namely all the combinations $(\pm \frac{1}{\sqrt{3}}, \pm \frac{1}{\sqrt{3}})$. To determine their nature we must study the spectral decomposition of the Hessian matrix

\[
Hf(x,y) = \begin{bmatrix}
6x & 0 \\
0 & 6y
\end{bmatrix}
\]

The Hessian has trivial eigenvalues $6x$ and $6y$. Accordingly, the points $(\frac{1}{\sqrt{3}}, -\frac{1}{\sqrt{3}})$ and $(-\frac{1}{\sqrt{3}}, \frac{1}{\sqrt{3}})$ are saddle points, $(-\frac{1}{\sqrt{3}}, -\frac{1}{\sqrt{3}})$ is a local maximum and $(\frac{1}{\sqrt{3}}, \frac{1}{\sqrt{3}})$ is a local minimum.




\end{enumerate}

\end{document}


